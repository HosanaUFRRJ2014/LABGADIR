\section{trab2\+\_\+2014780348.\+c}
\label{trab2__2014780348_8c_source}\index{grafalg16.\+2/trabs/autor\+\_\+2014780348/trab2\+\_\+2014780348.\+c@{grafalg16.\+2/trabs/autor\+\_\+2014780348/trab2\+\_\+2014780348.\+c}}

\begin{DoxyCode}
00001 \textcolor{preprocessor}{#include <graph.h>}
00002 \textcolor{preprocessor}{#include <heap.h>}
00003 
00004 \textcolor{keywordtype}{unsigned} \textcolor{keywordtype}{int} ident2() \{
00005         \textcolor{comment}{/*------Nome: Lívia de Azevedo da Silva-----*/}
00006         \textcolor{keywordflow}{return} 2014780348;
00007 \}
00008 
00009 \textcolor{keywordtype}{int} eh_subaciclico(Graph * \textcolor{keyword}{const} g, BitMap * \textcolor{keyword}{const} s) \{
00010         Graph *grafo\_S = newGraph(cardOf(s));
00011         Node nodeS,node,node\_viz;
00012     \textcolor{keywordtype}{int} mapeamentoParaS[cardOf(s)], i = 0,j = 0, vMap1, vMap2;
00013 
00014     \textcolor{comment}{//-------------Realizar o mapeamento dos vérticos do novo grafo-------------}
00015 
00016     \textcolor{keywordflow}{for}(begin(s,&nodeS); !end(&nodeS); next(&nodeS))
00017     \{
00018         \textcolor{comment}{/*O índice 0 corresponde ao 1ºElemento de S(índice do vértice em G), e assim por diante.}
00019 \textcolor{comment}{        Exemplo: S = \{23,2,33,10,5\}}
00020 \textcolor{comment}{        mapeamentoParaS = [23,2,33,10,5] -----> mapeamentoParaS[0] = 23; mapeamentoParaS[1] = 2; (...)*/}
00021         mapeamentoParaS[i] = getElement(&nodeS);
00022         i++;
00023     \}
00024 
00025     \textcolor{comment}{//--------------------------------------------------------------------------}
00026 
00027     \textcolor{comment}{//Construir a estrutura grafo para o subconjunto s de parâmetro.}
00028         \textcolor{keywordflow}{for}(begin(s,&nodeS), j = 0; !end(&nodeS); next(&nodeS),j++)
00029         \{
00030                 \textcolor{keywordflow}{for}(begin(neig(g,getElement(&nodeS)),&node\_viz); !end(&node\_viz); 
      next(&node\_viz))
00031                 \{
00032             vMap1 = j;
00033 
00034                         \textcolor{keywordflow}{if}(hasElement(s, getElement(&node\_viz)))
00035                         \{
00036                 \textcolor{keywordflow}{for}(i = 0;i < cardOf(s);i++)
00037                 \{
00038                     \textcolor{keywordflow}{if}(mapeamentoParaS[i] == getElement(&node\_viz))
00039                     \{
00040                         vMap2 = i;
00041                         \textcolor{keywordflow}{break};
00042                     \}
00043                 \}
00044 
00045                                 addEdge(grafo\_S, vMap1, vMap2);
00046                         \}
00047                 \}
00048         \}
00049 
00050         \textcolor{keywordflow}{return} eh_aciclico(grafo\_S);
00051 \}
00052 
00053 \textcolor{keywordtype}{int} eh_aciclico(Graph * \textcolor{keyword}{const} g) \{
00054         \textcolor{keywordtype}{int} n = getN(g), i = 0, w, v; \textcolor{comment}{//"d" é a quantidade de vizinhos de cada vértice.}
00055         BitMap *candidatos = newBitMap(getN(g));
00056         BitMap *a;
00057         Node node;
00058     \textcolor{keywordtype}{int} d[n], vertices[n];
00059 
00060     \textcolor{comment}{/*--------------Função de parâmetro para a heap.-------------*/}
00061     \textcolor{keywordtype}{int} comparar(\textcolor{keyword}{const} \textcolor{keywordtype}{void} *a, \textcolor{keyword}{const} \textcolor{keywordtype}{void} *b) 
00062     \{
00063         \textcolor{keywordflow}{return} d[*((\textcolor{keyword}{const} \textcolor{keywordtype}{int} *) a)] - d[*((\textcolor{keyword}{const} \textcolor{keywordtype}{int} *) b)];
00064     \}
00065     \textcolor{comment}{/*-------------------------------------------------------------*/}
00066 
00067         \textcolor{keywordflow}{for}(i = 0;i < n;i++)
00068         \{
00069                 d[i] = cardOf(neig(g, i));
00070         vertices[i] = i;
00071 
00072                 \textcolor{comment}{//Construindo o BitMap com os vértices do grafo.}
00073         \textcolor{comment}{//Adiciona todos os elementos do segundo bitmap no primeiro(lembrando que bitmap não permite
       elementos repetidos)}
00074                 addAll(candidatos,neig(g,i));
00075         \}
00076 
00077     heapify((\textcolor{keywordtype}{void}*)vertices, n, \textcolor{keyword}{sizeof}(\textcolor{keywordtype}{int}), comparar);
00078 
00079         \textcolor{keywordflow}{while}(n > 0 && d[vertices[0]] <= 1)
00080         \{
00081                 v = vertices[0];
00082         heappoll((\textcolor{keywordtype}{void}*)vertices, n, \textcolor{keyword}{sizeof}(\textcolor{keywordtype}{int}), comparar);
00083                 n -= 1;
00084 
00085                 \textcolor{keywordflow}{if}(d[v] == 1)
00086                 \{
00087                         a = cloneBitMap(candidatos);
00088                         intersectOf(a,a, neig(g,v));
00089                         begin(a, &node);
00090                         w = getElement(&node);
00091                         d[w] -= 1;
00092             heapup((\textcolor{keywordtype}{void}*)vertices, w, \textcolor{keyword}{sizeof}(\textcolor{keywordtype}{int}), comparar);
00093                 \}
00094         \}
00095 
00096         \textcolor{keywordflow}{if}(n == 0)
00097                 \textcolor{keywordflow}{return} 1;
00098 
00099         \textcolor{keywordflow}{return} 0;
00100 \}
\end{DoxyCode}
