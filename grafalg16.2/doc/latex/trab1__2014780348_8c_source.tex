\section{trab1\+\_\+2014780348.\+c}
\label{trab1__2014780348_8c_source}\index{grafalg16.\+2/trabs/autor\+\_\+2014780348/trab1\+\_\+2014780348.\+c@{grafalg16.\+2/trabs/autor\+\_\+2014780348/trab1\+\_\+2014780348.\+c}}

\begin{DoxyCode}
00001 \textcolor{preprocessor}{#include <graph.h>}
00002 
00003 \textcolor{keywordtype}{unsigned} \textcolor{keywordtype}{int} ident1() \{
00004         \textcolor{comment}{/*------Nome: Lívia de Azevedo da Silva-----*/}
00005         \textcolor{keywordflow}{return} 2014780348;
00006 \}
00007 
00008 BitMap * viz_comum(Graph * \textcolor{keyword}{const} g, \textcolor{keywordtype}{unsigned} \textcolor{keywordtype}{int} u, \textcolor{keywordtype}{unsigned} \textcolor{keywordtype}{int} v) \{
00009         BitMap * viz = newBitMap(getN(g));
00010         intersectOf(viz, neig(g,u), neig(g,v)); \textcolor{comment}{//Resultado da interseção entre os segundo e terceiro
       parâmetro e salva o resultado no 1º}
00011         \textcolor{keywordflow}{return} viz;
00012 \}
00013 
00014 \textcolor{keywordtype}{int} eh_clique(Graph * \textcolor{keyword}{const} g, BitMap * \textcolor{keyword}{const} s) \{
00015         BitMap * aux = cloneBitMap(s);
00016 
00017         Node node;
00018         \textcolor{keywordflow}{for} (begin(s, &node); !end(&node); next(&node)) \{
00019                 delElement(aux, &node);
00020                 \textcolor{keywordflow}{if} (!isSubset(aux, neig(g, getElement(&node))))
00021                         \textcolor{keywordflow}{return} 0;
00022         \}
00023 
00024         freeBitMap(aux);
00025         \textcolor{keywordflow}{return} 1;
00026 \}
00027 
00028 \textcolor{keywordtype}{int} viz_comum_eh_clique(Graph * \textcolor{keyword}{const} g, \textcolor{keywordtype}{unsigned} \textcolor{keywordtype}{int} u, \textcolor{keywordtype}{unsigned} \textcolor{keywordtype}{int} v) \{
00029         BitMap * viz = viz_comum(g, u, v);
00030         \textcolor{keywordtype}{int} ret = eh_clique(g, viz);
00031         freeBitMap(viz);
00032         \textcolor{keywordflow}{return} ret;
00033 \}
00034 
00035 \textcolor{keywordtype}{void} subgraus(Graph * \textcolor{keyword}{const} g, BitMap * \textcolor{keyword}{const} s, \textcolor{keywordtype}{unsigned} \textcolor{keywordtype}{int} * d) \{
00036         Node node;
00037         BitMap * aux = similarBitMap(s);
00038         \textcolor{keywordflow}{for} (begin(s, &node); !end(&node); next(&node)) \{
00039                 \textcolor{keywordtype}{unsigned} \textcolor{keywordtype}{int} v = getElement(&node);
00040                 intersectOf(aux, s, neig(g, v));
00041                 d[v] = cardOf(aux);
00042         \}
00043         freeBitMap(aux);
00044 \}
00045 
00046 \textcolor{keywordtype}{int} conta_arestas(Graph * \textcolor{keyword}{const} g, BitMap * \textcolor{keyword}{const} s) \{
00047         \textcolor{keywordtype}{int} qtd\_arestas = 0;
00048         BitMap *aux = newBitMap(getN(g)); \textcolor{comment}{//Estrutura de Dados com o tamanho de G}
00049         BitMap *auxS = cloneBitMap(s); \textcolor{comment}{// Cópia para uso no algoritmo do subconjunto S}
00050         Node node; \textcolor{comment}{//Nó auxiliar}
00051 
00052         \textcolor{keywordflow}{for} (begin(s, &node); !end(&node); next(&node)) 
00053         \{
00054                 \textcolor{comment}{/*Para não contar arestas repetidas, devemos retirar o vértice considerado da iteração para
       a contagem.}
00055 \textcolor{comment}{                Além disso, queremos descobrir apenas os vizinhos desse vértice e portanto deve-se
       retirá-lo do subconjunto }
00056 \textcolor{comment}{                para que o resultado da interseção seja apenas os seus vizinhos.*/}
00057                 delElement(auxS, &node);
00058                 intersectOf(aux,neig(g, getElement(&node)), auxS);
00059 
00060                 \textcolor{comment}{/*"cardOf" retorna a quantidade de elementos do conjunto de parâmetro.}
00061 \textcolor{comment}{                O número de arestas aqui equivale a quantidade de vizinhos em S do vértice considerado na
       iteração*/}
00062                 qtd\_arestas += cardOf(aux);
00063         \}       
00064 
00065         freeBitMap(aux);
00066         freeBitMap(auxS);
00067         \textcolor{keywordflow}{return} qtd\_arestas;
00068 \}
00069 
00070 \textcolor{keywordtype}{int} viz_comum_conta_arestas(Graph * \textcolor{keyword}{const} g, \textcolor{keywordtype}{unsigned} \textcolor{keywordtype}{int} u, \textcolor{keywordtype}{unsigned} \textcolor{keywordtype}{int} v) \{
00071         \textcolor{keywordflow}{return} conta_arestas(g, viz_comum(g, u, v)); \textcolor{comment}{//Uso da função "conta\_aresta" para a contagem nesse
       conjunto.}
00072 \}
00073 
00074 BitMap * grau_impar(Graph * \textcolor{keyword}{const} g, BitMap * \textcolor{keyword}{const} s) \{        
00075         BitMap *vertices\_grau\_impar = newBitMap(cardOf(s)); \textcolor{comment}{//Cria uma Estrutura de Dados do tamanho do
       subconjunto.}
00076         BitMap *auxS = cloneBitMap(s); \textcolor{comment}{//Cópia do subconjunto.}
00077         BitMap *aux = newBitMap(getN(g)); \textcolor{comment}{//Estrutura de Dados com o tamanho de G.}
00078         Node node; \textcolor{comment}{//Nó auxiliar.}
00079         \textcolor{keywordtype}{int} grau\_vertice; \textcolor{comment}{// grau do vértice correspondente da iteração.}
00080 
00081         \textcolor{keywordflow}{for} (begin(s,&node); !end(&node); next(&node))
00082         \{
00083                 \textcolor{comment}{/*-----------Definindo o conjunto dos vizinhos do alvo(node)(vértice da
       iteração).------------------------}
00084 \textcolor{comment}{                Com isso, conseguimos definir o grau desse vértice, que corresponde a quantidade de
       vizinhos que o mesmo possui.*/}
00085                 delElement(auxS, &node);
00086                 intersectOf(aux,neig(g, getElement(&node)), auxS);
00087                 grau\_vertice = cardOf(aux);
00088                 \textcolor{comment}{/*
      ------------------------------------------------------------------------------------------------------*/}
00089                 
00090                 \textcolor{comment}{/*Verificação se o grau do vértice é ímpar ou não*/}
00091                 \textcolor{keywordflow}{if} (grau\_vertice % 2 != 0)
00092                         addElement(vertices\_grau\_impar, &node); \textcolor{comment}{//Adiciona o vértice na solução.}
00093 
00094                 \textcolor{comment}{/*Adiciona novamente o vértice que foi retirado. Necessária adição, pois ele precisa ser
       contabilizado }
00095 \textcolor{comment}{                como vizinho em relação aos seus vizinhos.*/}
00096                 addElement(auxS, &node);
00097         \}
00098 
00099         freeBitMap(aux);
00100         freeBitMap(auxS);
00101         \textcolor{keywordflow}{return} cloneBitMap(vertices\_grau\_impar);        
00102 \}
\end{DoxyCode}
