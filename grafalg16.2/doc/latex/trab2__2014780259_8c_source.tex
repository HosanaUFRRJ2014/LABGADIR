\section{trab2\+\_\+2014780259.\+c}
\label{trab2__2014780259_8c_source}\index{grafalg16.\+2/trabs/autor\+\_\+2014780259/trab2\+\_\+2014780259.\+c@{grafalg16.\+2/trabs/autor\+\_\+2014780259/trab2\+\_\+2014780259.\+c}}

\begin{DoxyCode}
00001 \textcolor{preprocessor}{#include <graph.h>}
00002 \textcolor{preprocessor}{#include <heap.h>}
00003 
00004 \textcolor{keywordtype}{unsigned} \textcolor{keywordtype}{int} ident2() \{
00005         \textcolor{keywordflow}{return} 2014780259;
00006 \}
00007 
00008 \textcolor{keywordtype}{int} hash (\textcolor{keyword}{const} \textcolor{keywordtype}{void} * elemento)\{
00009     \textcolor{keywordflow}{return} (*(\textcolor{keyword}{const} \textcolor{keywordtype}{int}*) elemento);
00010 \}
00011 
00012 \textcolor{keywordtype}{int} eh_aciclico(Graph *\textcolor{keyword}{const} g) \{
00013     \textcolor{keywordtype}{int} tam = getN(g); \textcolor{comment}{// tam do grafo}
00014     \textcolor{keywordtype}{int} i = 0, graus[tam], alvos[tam], indices[tam], w, j;
00015     BitMap * aux = newBitMap(getN(g)); \textcolor{comment}{// bitmap auxiliar de tamanho do grafo}
00016     Node node;
00017 
00018     \textcolor{keywordtype}{int} Comparador(\textcolor{keyword}{const} \textcolor{keywordtype}{void} *a, \textcolor{keyword}{const} \textcolor{keywordtype}{void} *b) \{
00019         \textcolor{keywordflow}{return} graus[*((\textcolor{keyword}{const} \textcolor{keywordtype}{int} *) a)]-graus[*((\textcolor{keyword}{const} \textcolor{keywordtype}{int} *) b)];
00020     \}
00021 
00022     \textcolor{keywordflow}{for}(i=0;i<tam;i++)\{ \textcolor{comment}{// preenche todos os vetores. graus, alvos e indices}
00023         graus[i] = cardOf(neig(g, i));
00024         alvos[i] = i;
00025         indices[i] = i;
00026         setElement(&node,i);
00027         addElement(aux, &node); \textcolor{comment}{// Adiciona o elemento no bmp auxiliar. }
00028     \}    
00029 
00030     heapify_a(alvos, tam, \textcolor{keyword}{sizeof}(\textcolor{keywordtype}{int}), Comparador, indices, hash); \textcolor{comment}{// transforma o vetor alvos em uma heap}
00031 
00032     \textcolor{keywordflow}{while}((tam > 0) && (graus[alvos[0]] <= 1))\{    \textcolor{comment}{// condição do pseudocodigo lista 6}
00033         \textcolor{keywordtype}{int} alvo\_r = alvos[0], grau\_r = graus[alvos[0]]; \textcolor{comment}{// alvo removido e grau removido.}
00034 
00035         heappoll_a(alvos, tam, \textcolor{keyword}{sizeof}(\textcolor{keywordtype}{int}), Comparador, indices, hash); \textcolor{comment}{// remove a cabeça da heap alvos}
00036 
00037         tam--; \textcolor{comment}{// diminui o tamanho, pois a cabeça da heap foi removida}
00038 
00039         \textcolor{keywordflow}{if}(grau\_r == 1)\{ \textcolor{comment}{// se o grau removido = 1}
00040             BitMap * a = cloneBitMap(aux); 
00041             intersectOf(a, a, neig(g, alvo\_r)); \textcolor{comment}{// reter da lista}
00042             begin(a, &node); \textcolor{comment}{// partida no bitmap a que é clone do aux preenchido no primeiro for}
00043             graus[getElement(&node)] = graus[getElement(&node)] - 1; \textcolor{comment}{// atualiza os graus dos outros nós!}
00044             \textcolor{keywordflow}{for} (j = 0; j < getN(g); j++)
00045             \{
00046                 \textcolor{keywordflow}{if}(alvos[j] == getElement(&node))\{
00047                     w = i;
00048                 \}
00049             \}
00050             heapup_a(alvos, w, \textcolor{keyword}{sizeof}(\textcolor{keywordtype}{int}), Comparador, indices, hash);
00051         \}
00052     \}
00053     \textcolor{keywordflow}{if}(tam == 0)\{
00054         \textcolor{keywordflow}{return} 1;
00055     \}
00056 
00057     \textcolor{keywordflow}{return} 0;
00058 \}
00059 
00060 \textcolor{keywordtype}{int} eh_subaciclico(Graph * \textcolor{keyword}{const} g, BitMap * \textcolor{keyword}{const} s) \{
00061     Graph *a = newGraph(getN(g)), *b = newGraph(cardOf(s));
00062     Node node;
00063     Node node2;
00064 
00065     \textcolor{keywordtype}{int} alvosBmp[cardOf(s)];
00066     
00067     
00068     \textcolor{keywordtype}{int} alvoVisitado1, alvoVisitado2, i=0;
00069 
00070     \textcolor{keywordflow}{for} (begin(s, &node); !end(&node); next(&node))\{ \textcolor{comment}{//Adicionando alvos no alvosBmp[i]}
00071         alvosBmp[i] = getElement(&node);
00072         i++;
00073     \}
00074 
00075     \textcolor{keywordflow}{for} (begin(s, &node); !end(&node); next(&node)) \{ \textcolor{comment}{// Percorre o subconjunto do bitmap}
00076         \textcolor{keywordflow}{for} (begin(neig(g, getElement(&node)), &node2); !end(&node2); next(&node2)) \{ \textcolor{comment}{// Percorre conjunto
       de vizinhos do nó de cima no nó 2}
00077             \textcolor{keywordflow}{if} (hasEdge(a, getElement(&node), getElement(&node2)) == 0 && 
      hasElement(s, getElement(&node2))) \{ \textcolor{comment}{// se não existir aresta entre no1 e no2 E no2 existir}
00078                 addEdge(a, getElement(&node), getElement(&node2)); \textcolor{comment}{// Adiciona aresta no 1 e no 2}
00079                 \textcolor{keywordflow}{for}(i = 0; i < cardOf(s); i++)\{ \textcolor{comment}{// mapeia restante dos vértices}
00080                     \textcolor{keywordflow}{if}(alvosBmp[i] == getElement(&node))\{
00081                         alvoVisitado1=i;
00082                     \}
00083                     \textcolor{keywordflow}{if}(alvosBmp[i] == getElement(&node2))\{
00084                         alvoVisitado2=i;
00085                     \}
00086                 \}
00087                 addEdge(b, alvoVisitado1, alvoVisitado2);
00088             \}
00089         \}
00090     \}
00091     \textcolor{keywordflow}{return} eh_aciclico(b);
00092 \} 
\end{DoxyCode}
