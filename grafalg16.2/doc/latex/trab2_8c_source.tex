\section{trab2.\+c}
\label{trab2_8c_source}\index{grafalg16.\+2/trabs/trab2/trab2.\+c@{grafalg16.\+2/trabs/trab2/trab2.\+c}}

\begin{DoxyCode}
00001 \textcolor{preprocessor}{#include <trab2.h>}
00002 
00003 \textcolor{preprocessor}{#include <stdio.h>}
00004 \textcolor{preprocessor}{#include <trab_main.h>}
00005 \textcolor{preprocessor}{#include <graph.h>}
00006 \textcolor{preprocessor}{#include <heap.h>}
00007 \textcolor{preprocessor}{#include <trab1.h>}
00008 
00009 \textcolor{keyword}{typedef} \textcolor{keyword}{enum} \{ NULO=-1, EXSUBACICLICO, EHSUBACICLICO, EHACICLICO, NPROBS2 \} PROBS2;
00010 
00011 \textcolor{keyword}{static} \textcolor{keyword}{const} \textcolor{keywordtype}{string} nomes[NPROBS2] = \{  \textcolor{stringliteral}{"Exemplo de execução de algoritmo de verificação de subgrafo
       acíclico"},
00012                                                                                 \textcolor{stringliteral}{"Verificação de subgrafo
       acíclico"},
00013                                                                                 \textcolor{stringliteral}{"Verificação de grafo
       acíclico"} \};
00014 \textcolor{keyword}{static} \textcolor{keyword}{const} \textcolor{keywordtype}{string} descricoes[NPROBS2] = \{ \textcolor{stringliteral}{"Exemplo de execução de um algoritmo para o problema de, dados
       um grafo G e um subconjunto s dos seus vértices, determinar se s define um subgrafo acíclico de G. O
       algoritmo é baseado em sucessivas remoções de vértices de grau menor que 2."},
00015                                                                                         \textcolor{stringliteral}{"Dados um grafo G e
       um subconjunto s dos seus vértices, determinar se s define um subgrafo acíclico de G."},
00016                                                                                         \textcolor{stringliteral}{"Dado um grafo G,
       determinar se G é acíclico."} \};
00017 
00018 \textcolor{keyword}{static} \textcolor{keyword}{const} Trab \_trab2 = \{ NPROBS2, \textcolor{stringliteral}{"Trabalho Prático sobre Grafos"}, nomes, descricoes \};
00019 
00020 Trab \textcolor{keyword}{const} * descTrab2() \{
00021         \textcolor{keywordflow}{return} &\_trab2;
00022 \}
00023 
00024 \textcolor{keyword}{static} \textcolor{keywordtype}{int} hash(\textcolor{keyword}{const} \textcolor{keywordtype}{void} *i) \{
00025         \textcolor{keywordflow}{return} *((\textcolor{keyword}{const} \textcolor{keywordtype}{int} *) i);
00026 \}
00027 
00028 \textcolor{keyword}{static} \textcolor{keywordtype}{int} eh\_subaciclico\_prof(Graph * \textcolor{keyword}{const} g, BitMap * \textcolor{keyword}{const} s, \textcolor{keywordtype}{int} verb) \{
00029         BitMap * cands = cloneBitMap(s);
00030         \textcolor{keywordtype}{unsigned} \textcolor{keywordtype}{int} nc = cardOf(cands);
00031         \textcolor{keywordtype}{unsigned} \textcolor{keywordtype}{int} h[nc];
00032         \textcolor{keywordtype}{unsigned} \textcolor{keywordtype}{int} map[getN(g)];
00033         \textcolor{keywordtype}{unsigned} \textcolor{keywordtype}{int} d[getN(g)];
00034 
00035         subgraus(g, s, d);
00036         Node node;
00037         \textcolor{keywordtype}{int} i = 0;
00038         \textcolor{keywordflow}{for} (begin(cands, &node); !end(&node); next(&node)) \{
00039                 h[i] = getElement(&node);
00040                 map[h[i]] = i++;
00041         \}
00042 
00043         \textcolor{keywordtype}{int} compard(\textcolor{keyword}{const} \textcolor{keywordtype}{void} *i, \textcolor{keyword}{const} \textcolor{keywordtype}{void} *j) \{
00044                 \textcolor{keywordflow}{return} d[*((\textcolor{keyword}{const} \textcolor{keywordtype}{int} *) i)] - d[*((\textcolor{keyword}{const} \textcolor{keywordtype}{int} *) j)];
00045         \}
00046 
00047         heapify_a(h, nc, \textcolor{keyword}{sizeof}(\textcolor{keywordtype}{unsigned} \textcolor{keywordtype}{int}), compard, map, hash);
00048 
00049         \textcolor{keywordflow}{if} (verb) \{
00050                 printf(\textcolor{stringliteral}{"Graus \(\backslash\)t\(\backslash\)t ["});
00051                 \textcolor{keywordflow}{for} (i = 0; i < nc; i++)
00052                         printf(\textcolor{stringliteral}{" %u"},d[h[i]]);
00053                 printf(\textcolor{stringliteral}{" ]\(\backslash\)nCandidatos \(\backslash\)t ["});
00054                 \textcolor{keywordflow}{for} (i = 0; i < nc; i++)
00055                         printf(\textcolor{stringliteral}{" %u"},h[i]);
00056                 printf(\textcolor{stringliteral}{" ]\(\backslash\)n"});
00057         \}
00058 
00059         BitMap * aux = similarBitMap(s);
00060         \textcolor{keywordflow}{while} (nc > 0 && d[h[0]] <= 1) \{
00061                 \textcolor{keywordtype}{int} v = -1;
00062                 \textcolor{keywordflow}{if} (d[h[0]] == 1) \{
00063                         intersectOf(aux, cands, neig(g, h[0]));
00064                         begin(aux, &node);
00065                         v = getElement(&node);
00066                 \}
00067 
00068                 setElement(&node, h[0]);
00069                 delElement(cands, &node);
00070                 heappoll_a(h, nc, \textcolor{keyword}{sizeof}(\textcolor{keywordtype}{unsigned} \textcolor{keywordtype}{int}), compard, map, hash);
00071                 nc--;
00072 
00073                 \textcolor{keywordflow}{if} (v >= 0) \{
00074                         d[v]--;
00075                         heapup_a(h, map[v], \textcolor{keyword}{sizeof}(\textcolor{keywordtype}{unsigned} \textcolor{keywordtype}{int}), compard, map, hash);
00076                 \}
00077 
00078                 \textcolor{keywordflow}{if} (verb) \{
00079                         printf(\textcolor{stringliteral}{"\(\backslash\)nGraus \(\backslash\)t\(\backslash\)t ["});
00080                         \textcolor{keywordflow}{for} (i = 0; i < nc; i++)
00081                                 printf(\textcolor{stringliteral}{" %u"},d[h[i]]);
00082                         printf(\textcolor{stringliteral}{" ]\(\backslash\)nCandidatos \(\backslash\)t ["});
00083                         \textcolor{keywordflow}{for} (i = 0; i < nc; i++)
00084                                 printf(\textcolor{stringliteral}{" %u"},h[i]);
00085                         printf(\textcolor{stringliteral}{" ]\(\backslash\)n"});
00086                 \}
00087         \}
00088         freeBitMap(aux);
00089         freeBitMap(cands);
00090 
00091         \textcolor{keywordflow}{return} nc == 0;
00092 \}
00093 
00094 \textcolor{keyword}{static} \textcolor{keywordtype}{int} eh\_aciclico\_prof(Graph * \textcolor{keyword}{const} g) \{
00095         \textcolor{keywordflow}{return} 0;
00096 \}
00097 
00098 \textcolor{keyword}{static} \textcolor{keywordtype}{void} entradaIntVet(FILE * stream, \textcolor{keywordtype}{int} ** v, \textcolor{keywordtype}{unsigned} \textcolor{keywordtype}{int} * n) \{
00099         \textcolor{keywordflow}{if} (stream == stdin) printf(\textcolor{stringliteral}{"Entre o tamanho do vetor: "});
00100         fscanf(stream,\textcolor{stringliteral}{"%u"},n);
00101         *v = (\textcolor{keywordtype}{int} *) calloc(*n, \textcolor{keyword}{sizeof}(\textcolor{keywordtype}{int}));
00102         \textcolor{keywordtype}{int} i;
00103         \textcolor{keywordflow}{for} (i = 0; i < *n; i++) \{
00104                 \textcolor{keywordflow}{if} (stream == stdin) printf(\textcolor{stringliteral}{"Entre o elemento v[%d]: "},i);
00105                 fscanf(stream,\textcolor{stringliteral}{"%d"},&(*v)[i]);
00106         \}
00107 \}
00108 
00109 \textcolor{keywordtype}{unsigned} \textcolor{keywordtype}{int} execTrab2(Graph * g, FILE *stream, \textcolor{keywordtype}{int} prob) \{
00110         \textcolor{keywordtype}{unsigned} \textcolor{keywordtype}{int} ret = 0;
00111 
00112         \textcolor{keywordflow}{switch} (prob) \{
00113         \textcolor{keywordflow}{case} EXSUBACICLICO: \{
00114                 \textcolor{keywordtype}{int} * v;
00115                 \textcolor{keywordtype}{unsigned} \textcolor{keywordtype}{int} n;
00116                 entradaIntVet(stream, &v, &n);
00117 
00118                 BitMap * s\_prof = pack(v, n, getN(g));
00119                 ret = eh\_subaciclico\_prof(g, s\_prof, 1);
00120                 freeBitMap(s\_prof);
00121                 free(v);
00122 
00123                 ret = 1;
00124                 \textcolor{keywordflow}{break};
00125         \}
00126         \textcolor{keywordflow}{case} EHSUBACICLICO: \{
00127                 \textcolor{keywordtype}{int} * v;
00128                 \textcolor{keywordtype}{unsigned} \textcolor{keywordtype}{int} n;
00129                 entradaIntVet(stream, &v, &n);
00130 
00131                 BitMap * s = pack(v, n, getN(g));
00132                 \textcolor{keywordtype}{int} ec = eh_subaciclico(g, s);
00133                 \textcolor{keywordflow}{if} (stream == stdin) printf(\textcolor{stringliteral}{"Resultado obtido: %d\(\backslash\)n"}, ec);
00134 
00135                 BitMap * s\_prof = pack(v, n, getN(g));
00136                 ret = ((!ec == !eh\_subaciclico\_prof(g, s\_prof, 0)) && areEqual(s, s\_prof));
00137 
00138                 freeBitMap(s);
00139                 freeBitMap(s\_prof);
00140                 free(v);
00141                 \textcolor{keywordflow}{break};
00142         \}
00143         \textcolor{keywordflow}{case} EHACICLICO: \{
00144                 \textcolor{keywordtype}{int} ec = eh_aciclico(g);
00145                 \textcolor{keywordflow}{if} (stream == stdin) printf(\textcolor{stringliteral}{"Resultado obtido: %d\(\backslash\)n"}, ec);
00146 
00147                 ret = (!ec == !eh\_aciclico\_prof(g));
00148 
00149                 \textcolor{keywordflow}{break};
00150         \}
00151         \textcolor{keywordflow}{default}:
00152                 \textcolor{keywordflow}{break};
00153         \}
00154 
00155         \textcolor{keywordflow}{return} ret;
00156 \}
\end{DoxyCode}
